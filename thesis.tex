%% Use the RPI thesis document class.
\documentclass[chap]{rpi_thesis}

% Provides 1 in. margins on all sides of the page.
% Adding the option 'showframe' to the \geometry options below
% will show an outline of the margins on all pages.
\usepackage{geometry}
\geometry{
left=1in,
right=1in,
top=1in,
bottom=1in}

% These options will provide most of the options for an IEEE
% compliant bibliography and captions for figures. It is up to users to
% capitilizations / abbreviations and other formatting is correct
% in the `references.bib' file.
\usepackage{cite}
\usepackage{notoccite}
\bibliographystyle{rpi_IEEE}
\renewcommand{\bibname}{}

% Consistently caption figures and tables  with a period
% separating the figure number and the figure caption and
% comply with the IEEE style guide for figure options.
\usepackage[
  labelsep=period,
  justification=centering]{caption}
\renewcommand{\captionfont}{\bfseries}
\renewcommand{\figurename}{Fig.}

% Place captions above tables as required by OGE
\usepackage{float}
\floatstyle{plaintop}
\restylefloat{table}

% This package allows for the easy inclusion of figures
\usepackage{graphicx}

% This package allows for hyperrefs in the document, which can
% allow easy navigation to citations and referenced equations,
% chapters, and sections. This is very useful when writing.
\usepackage[hidelinks]{hyperref}

% The following packages are used to generate generic text and
% can be removed once editing begins.
\usepackage[english]{babel}
\usepackage{blindtext}

% The main body of the thesis document. Add or remove files as needed.
\begin{document}
\thesistitle{\bf MY SUPER GREAT THESIS TITLE}

\author{A Great Graduate Student}
\degree{Doctor of Philosophy}
\department{Mechanical, Aerospace, and Nuclear Engineering}
\submitdate{[May 2018] \\ Submitted January 2018}
\copyrightyear{2018}

\titlepage
\copyrightpage
\tableofcontents
\listoftables
\listoffigures

\specialhead{ACKNOWLEDGMENT}

Thanks to everyone. I'll buy you all ice cream.

\specialhead{ABSTRACT}

This is super important work that is super
important.
This is super important work that is super
important.
This is super important work that is super
important.
This is super important work that is super
important.
This is super important work that is super
important.
This is super important work that is super
important.

This is super important work that is super
important.
This is super important work that is super
important.
This is super important work that is super
important.
This is super important work that is super
important.
This is super important work that is super
important.
This is super important work that is super
important.
This is super important work that is super
important.
This is super important work that is super
important.
This is super important work that is super
important.

\chapter{THE FIRST CHAPTER}
\label{chap:chapterone}

\let\thefootnote\relax\footnotetext{
This chapter has been submitted to:
M.Y. Name, ``My super great article,''
submitted for publication.}

\section{Introduction}

\blindtext

\section{Content}

\blindtext

\begin{table}[hbt!]
\centering
\begin{tabular}{|c | c |}
\hline
a & b \\ \hline\hline
0.01 & 0.02 \\ \hline
0.03 & 0.04 \\ \hline
\end{tabular}
\caption{An example of a table in the thesis document.
Notice the caption is above the table.}
\label{table:mytable}
\end{table}

\blindtext

\begin{figure}[ht!]
\centering
\includegraphics[width=0.25\textwidth]{rpi_seal}
\caption{An example of a figure in the thesis document.
Notice the caption is below the table.}
\label{fig:myfigure}
\end{figure}

Look! I am referencing Figure \ref{fig:myfigure}.
Look! I am referencing Table \ref{table:mytable}.
Look! I am referencing a citation \cite{areference}.

\section{Conclusions}

\blindtext

\include{bibliography}
\appendix
\addtocontents{toc}{\parindent0pt\vskip12pt APPENDICES}

\chapter{AN APPENDIX CHAPTER}
\blindtext

\end{document}
